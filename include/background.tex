\iffalse \bibliography{include/backmatter/references} \fi
\chapter{Background}

This section seeks to present the definitions and keywords used for the gathering of relevant literature. Keywords are aimed for relating to the research questions of the study.

\section{Leadership}

To fully understand the topic of this study, a clear definition of leadership is required. It is crucial to understand the definition of leadership as it is a word often related to a broad spectra of subjects. Leadership is not a single entity as it appears in many forms and takes place in an endless set of contexts therefore this research seeks to understand how leadership is perceived and materialised in the context of agile software development projects. A holistic presentation of the many definitions of leadership will be provided. As the scope of this study is limited, each of the definitions will only be presented thus only scratching the surface of what is available within the field of leadership.\\

\cite{book:441137} provides an aggregation of leadership definitions from multiple literature sources. The definitions are presented in list \ref{list:leader-def}. There exists a few notable similarities and keywords presented in list \ref{list:leader-def} for how literature defines leadership. Namely, \textit{\quotes{one or more individuals ability to influence, arouse, and satisfy the motives of others to achieve an organisational goal}}. Thus guiding this study towards leadership keywords such as: \texttt{influence}, \texttt{motivate}, \texttt{achieve}, \texttt{organisation}, \texttt{goal}, \texttt{followers}. It is of interest for this study to understand how individuals influence and motivate a software development team for a project utilising agile development paradigms.\\

\begin{mylist}[H]
\caption{Aggregated leadership definitions presented by \cite{book:441137}}
\centering
\label{list:leader-def}
	\begin{itemize}
		\item \textit{\quotes{The behaviour of an individual … directing the activities of a group towards a shared goal.}} \cite[p.7]{hemphill1957development}
		\item \textit{\quotes{The influential increment over and above mechanical compliance with the routine directives of the organisation.}} \cite[p.528]{katz1978social}
		\item \textit{\quotes{Leadership is exercised when persons … mobilize … institutional, political, psychological, and other resources so as to arouse, engage, and satisfy the motives of followers.}} \cite[p.18]{burns1978leadership}
		\item \textit{\quotes{Leadership is realized in the process whereby one or more individuals succeed in attempting to frame and define the reality of others.}} \cite[p.258]{smircich1982leadership}
		\item \textit{\quotes{The process of influencing the activities of an organized group toward goal achievement.}} \cite[p.46]{rauch1984functionalism}
		\item \textit{\quotes{Leadership is about articulating visions, embodying values, and creating the environment within which things can be accomplished.}} \cite[p.206]{richards1986after}
		\item \textit{\quotes{Leadership is a process of giving purpose (meaningful direction) to collective effort, and causing willing effort to be expended to achieve purpose.}} \cite[p.281]{jacobs1990military}
		\item \textit{\quotes{The ability to step outside the culture … to start evolutionary change processes that are more adaptive.}} \cite[p.2]{jacobs1990military}
		\item \textit{\quotes{Leadership is the process of making sense of what people are doing together so that people will understand and be committed.}} \cite[p.4]{drath1994making}
		\item \textit{\quotes{The ability of an individual to influence, motivate, and enable others to contribute toward the effectiveness and success of the organization…}} \cite[p.184]{house1999cultural}\\
	\end{itemize}
\end{mylist}
\clearpage

\section{Agile Software Development}

Furthermore, as an increasing amount software development paradigms evolve as predecessors to agile software development it becomes intriguing to understand how the leadership of such subsets materialise. This section seeks to clarify the research questions by providing the definition of agile software development, and further describe a number of popular software development paradigms within the field.\\

Within software development there exists two predominant methodologies, namely (i) traditional software engineering and (ii) agile software development. This study focuses the scope to agile software development. Compared to traditional software engineering, agile software development implements a dynamic, non-deterministic and non-linear iterative approach towards the method for developing software. Whereas traditional software development is characterised by its implementation of a sequential approach for developing software where the project takes the form of stages where each is required to finalise before moving on to the next (\cite{larman2004agile}).\\

Agile software development has become increasingly popular since its birth in the early 2000's. It aims to allow for delivering software fast where it relies on \textit{\quotes{people and their creativity rather than on processes}} (\cite{nerur2005challenges}; \cite{dyba2000improvisation}). Agile development acts as an umbrella for a number of approaches for developing software, i.e. scrum, lean, extreme programming (XP), crystal, among others. Furthermore, a member derived from agile software development is the methodology DevOps. Both share many of their philosophies derived from lean management, such as collaboration and communication. DevOps (development operations) was born from agile development when organisations began to see an increase of releases after adopting agile development. DevOps aims to institute an environment where software releasing is occurring faster and more reliable.\\

This research aims to understand the leadership of two dominant subsets of agile software development, namely: 1) Scrum, 2) DevOps.\\







